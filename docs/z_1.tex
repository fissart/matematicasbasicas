\PassOptionsToPackage{unicode=true}{hyperref} % options for packages loaded elsewhere
\PassOptionsToPackage{hyphens}{url}
\PassOptionsToPackage{dvipsnames,svgnames*,x11names*}{xcolor}
%
\documentclass[16pt,]{krantz}
\usepackage{lmodern}
\usepackage{amssymb,amsmath}
\usepackage{ifxetex,ifluatex}
\usepackage{fixltx2e} % provides \textsubscript
\ifnum 0\ifxetex 1\fi\ifluatex 1\fi=0 % if pdftex
  \usepackage[T1]{fontenc}
  \usepackage[utf8]{inputenc}
  \usepackage{textcomp} % provides euro and other symbols
\else % if luatex or xelatex
  \usepackage{unicode-math}
  \defaultfontfeatures{Ligatures=TeX,Scale=MatchLowercase}
    \setmonofont[Mapping=tex-ansi,Scale=0.7]{Source Code Pro}
\fi
% use upquote if available, for straight quotes in verbatim environments
\IfFileExists{upquote.sty}{\usepackage{upquote}}{}
% use microtype if available
\IfFileExists{microtype.sty}{%
\usepackage[]{microtype}
\UseMicrotypeSet[protrusion]{basicmath} % disable protrusion for tt fonts
}{}
\IfFileExists{parskip.sty}{%
\usepackage{parskip}
}{% else
\setlength{\parindent}{0pt}
\setlength{\parskip}{6pt plus 2pt minus 1pt}
}
\usepackage{xcolor}
\usepackage{hyperref}
\hypersetup{
            pdftitle={Matemáticas básicas},
            pdfauthor={Ricardo Michel MALLQUI BAÑOS},
            colorlinks=true,
            linkcolor=Maroon,
            filecolor=Maroon,
            citecolor=Blue,
            urlcolor=Blue,
            breaklinks=true}
\urlstyle{same}  % don't use monospace font for urls
\usepackage{longtable,booktabs}
% Fix footnotes in tables (requires footnote package)
\IfFileExists{footnote.sty}{\usepackage{footnote}\makesavenoteenv{longtable}}{}
\setlength{\emergencystretch}{3em}  % prevent overfull lines
\providecommand{\tightlist}{%
  \setlength{\itemsep}{0pt}\setlength{\parskip}{0pt}}
\setcounter{secnumdepth}{5}

% set default figure placement to htbp
\makeatletter
\def\fps@figure{htbp}
\makeatother

\usepackage[spanish,es-lcroman]{babel}
\usepackage{booktabs}
\usepackage{mathtools}
\usepackage{graphicx}
\usepackage{amsmath}
\usepackage{makeidx}
\makeindex
%\usepackage{showframe}
%\usepackage[a4paper]{geometry}
%\geometry{verbose,tmargin=3cm,bmargin=3cm,lmargin=3.5cm,rmargin=3cm}
\setlength\parindent{23pt}
\usepackage[singlelinecheck=off]{caption}

\usepackage{times}
\renewcommand{\rmdefault}{ptm}
%\usepackage[lite,subscriptcorrection,nofontinfo,zswash]{mtpro2}

\usepackage{graphicx}

% Determine if the image is too wide for the page.
\makeatletter
\def\ScaleIfNeeded{%
  \ifdim\Gin@nat@width>\linewidth
    \linewidth
  \else
    \Gin@nat@width
  \fi
}
\makeatother

% Resize figures that are too wide for the page.
%\let\oldincludegraphics\includegraphics
%\renewcommand\includegraphics[2][]{%
%  \oldincludegraphics[scale=0.85]{#2}
%}

\usepackage{amsthm}
\makeatletter
\def\thm@space@setup{%
  \thm@preskip=8pt plus 2pt minus 4pt
  \thm@postskip=\thm@preskip
}
\makeatother

\flushbottom 

\frontmatter
\usepackage[]{natbib}
\bibliographystyle{apalike}

\title{Matemáticas básicas}
\author{Ricardo Michel MALLQUI BAÑOS}
\providecommand{\institute}[1]{}
\institute{Universidad Nacional San Cristóbal De Huamanga}
\date{2021-07-12}

\usepackage{amsthm}
\newtheorem{theorem}{Teorema}[chapter]
\newtheorem{lemma}{Lema}[chapter]
\newtheorem{corollary}{Corolario}[chapter]
\newtheorem{proposition}{Proposición}[chapter]
\newtheorem{conjecture}{Conjectura}[chapter]
\theoremstyle{definition}
\newtheorem{definition}{Definición}[chapter]
\theoremstyle{definition}
\newtheorem{example}{Ejemplo}[chapter]
\theoremstyle{definition}
\newtheorem{exercise}{Ejercicio}[chapter]
\theoremstyle{definition}
\newtheorem{hypothesis}{Hypothesis}[chapter]
\theoremstyle{remark}
\newtheorem*{remark}{Observación}
\newtheorem*{solution}{Solución}
\begin{document}
\maketitle

%\cleardoublepage\newpage\thispagestyle{empty}\null
%\cleardoublepage\newpage\thispagestyle{empty}\null
%\cleardoublepage\newpage
\thispagestyle{empty}
\begin{flushright}
Universidad Nacional San Cristobal de Huamanga

Fisart.cf

Agradecimento a los estudiantes de la ESFAPA FGPA

A la UNSCH

\includegraphics[height=3cm]{U.pdf}
\end{flushright}
%\setlength{\abovedisplayskip}{-5pt}
%\setlength{\abovedisplayshortskip}{-5pt}

{
\hypersetup{linkcolor=}
\setcounter{tocdepth}{2}
\tableofcontents
}
\listoftables
\listoffigures
\newcommand{\N}{\mathbb{N}}
\newcommand{\R}{\mathbb{R}}
\newcommand{\CC}{\mathbb{C}}
\newcommand{\I}{\mathbb{I}}
\newcommand{\f}{\mathbb{f}}
\newcommand{\X}{\mathbb{X}}
\newcommand{\D}{\mathbb{D}}
\newcommand{\Z}{\mathbb{Z}}
\newcommand{\Q}{\mathbb{Q}}
\newcommand{\norm}[1]{\left\Vert#1\right\Vert}
\newcommand{\abs}[1]{\left\vert#1\right\vert}
\newcommand{\set}[1]{\left\{#1\right\}}
\newcommand{\seq}[1]{\left<#1\right>}
\newcommand{\co}[1]{\left[#1\right]}
\newcommand{\cc}[1]{\left(#1\right)}
\newcommand{\J}{\mathcal{J}}
\newcommand{\K}{\mathcal{K}}
\newcommand{\M}{\mathcal{M}}
\newcommand{\F}{\mathcal{F}}

\hypertarget{resumen}{%
\chapter*{Resumen}\label{resumen}}


www.

\hypertarget{introducciuxf3n}{%
\chapter*{Introducción}\label{introducciuxf3n}}


www.

\mainmatter

\hypertarget{logica}{%
\chapter{Logica}\label{logica}}

\begin{longtable}[]{@{}cccl@{}}
\caption{\label{tab:ww} Sed.}\tabularnewline
\toprule
\(p\) & \(q\) & \(w \rightarrow \left( \sim p \vee q\right)\) & \(w\)\tabularnewline
\midrule
\endfirsthead
\toprule
\(p\) & \(q\) & \(w \rightarrow \left( \sim p \vee q\right)\) & \(w\)\tabularnewline
\midrule
\endhead
V & V & w & w\tabularnewline
V & F & w & w\tabularnewline
F & V & w & w\tabularnewline
F & F & w & w\tabularnewline
\bottomrule
\end{longtable}

\[ \Longrightarrow \Longleftrightarrow  \]

\[ \Longleftrightarrow  \]
\[ \Longleftrightarrow  \]
\[ \Longleftrightarrow \vee \wedge \rightarrow \left( \int_{2}^{2} \right) \left[ \int_{3}^{2} \right]  \left\{ w,w,w,w,w,w,w/x \in W \right\}  \]

Refiérase al cuadro \ref{tab:regulares} \ref{tab:ww}

\begin{longtable}[]{@{}cccc@{}}
\caption{\label{tab:regulares} Polígonos cerrados regulares.}\tabularnewline
\toprule
Tipo & Número de lados & Número de diagonales & Apotemas\tabularnewline
\midrule
\endfirsthead
\toprule
Tipo & Número de lados & Número de diagonales & Apotemas\tabularnewline
\midrule
\endhead
Triángulo equilátero & 3 & 0 & 3\tabularnewline
Cuadrado & 4 & 2 & 4\tabularnewline
Pentágono & 5 & 5 & 5\tabularnewline
Exagono & 6 & 6 & 6\tabularnewline
Heptagono & 7 & 7 & 7\tabularnewline
\ldots{} & \ldots{} & \ldots{} & \ldots{}\tabularnewline
\bottomrule
\end{longtable}

\hypertarget{conjuntos}{%
\chapter{Conjuntos}\label{conjuntos}}

\begin{definition}[Conjunto]
\protect\hypertarget{def:conjunto}{}{\label{def:conjunto} \iffalse (Conjunto) \fi{} }Es una coleccion de elementos con caractersiticas similares
\end{definition}
\#\#\# Determincación de un conjunto

\begin{definition}[Determinacion de conjuntos]
\protect\hypertarget{def:conjunto2}{}{\label{def:conjunto2} \iffalse (Determinacion de conjuntos) \fi{} }Por extensión y comprensión
\end{definition}
* \textbf{Extensión}
\[A=\left\{ 1,2,3,4,5,6,7 \right\} \]
* \textbf{Comprensión}
\[A=\left\{ x \in \mathbb{N};0<x<5 \right\}  \]

\hypertarget{conjuntos-buxe1sicos}{%
\subsection{Conjuntos básicos}\label{conjuntos-buxe1sicos}}

Conjuntos universal, vacio, unitario
* \textbf{Conjunto de los sitemas numericos}
\[
\mathbb{N}, \mathbb{Z}, \mathbb{Q}, \mathbb{I}, \mathbb{R}, \mathbb{C}
\]
* \textbf{Cojunto vacio}
\[
\phi=\left\{x/x\neq x\right\}
\]
* \textbf{Conjunto unitario}
\[
A=\left\{a\right\}
\]

\hypertarget{funciuxf3n-proposicional-y-cuantificadores}{%
\section{Función proposicional y cuantificadores}\label{funciuxf3n-proposicional-y-cuantificadores}}

\hypertarget{funciuxf3n-proposicional}{%
\subsection{Función proposicional}\label{funciuxf3n-proposicional}}

\begin{definition}[Función proposicional]
\protect\hypertarget{def:proposicional}{}{\label{def:proposicional} \iffalse (Función proposicional) \fi{} }Sea \(x\) una variable \(P(x)\) un \emph{enunciado}, \(P(x)\) es una \textbf{\emph{función proposicional}} si al sustituir la variable con una constante este se convierte en una \emph{proposición}.
\end{definition}
Sea \(x\) una variable \(P(x)\) un \emph{enunciado}, \(P(x)\) es una \textbf{\emph{función proposicional}} si al sustituir la variable con una constante este se convierte en una \emph{proposición}.

Por ejemplo \(P(x)\): \(x\) es un numero par

Al conjunto de todos lo valores de \(x\) se denomina \emph{domino de la variable}

\hypertarget{cuantificadores}{%
\subsection{Cuantificadores}\label{cuantificadores}}

\begin{definition}[Cuantificador existencial]
\protect\hypertarget{def:existencial}{}{\label{def:existencial} \iffalse (Cuantificador existencial) \fi{} }\[\exists\] Es una generalización de la disyunción Inclusiva. Por ello, es verdadero cuando al menos un valor de \(x\) perteneciente al Dominio de \(A\), es Verdadero. Se denota; \(\exists x / P (x)\) Se lee: ``Existe al menos un \(x\)'', ``Algunos \(x\)''," Hay \(x\)``,''Existe un \(x\)``, etc.
\end{definition}
\[\exists\] Es una generalización de la disyunción Inclusiva. Por ello, es verdadero cuando al menos un valor de \(x\) perteneciente al Dominio de \(A\), es Verdadero. Se denota; \(\exists x / P (x)\) Se lee:''Existe al menos un \(x\)``,''Algunos \(x\)``,'' Hay \(x\)``,''Existe un \(x\)", etc.

\begin{definition}[Cuantificador universal]
\protect\hypertarget{def:universal}{}{\label{def:universal} \iffalse (Cuantificador universal) \fi{} }\[\forall\] Es una generalización de la \emph{conjunción}. Debido a esto es verdadero cuando todos los valores de \(x\) que pertenecen al Dominio de \(A\) son Verdaderos. Se denota: \(\forall x ; p(x)\) Se lee: ``Para Todo \(x\)'', ``Para cada \(x\)'', ``Todos (as) las \(x\)'', ``Todo \(x\)''.
\end{definition}
\[\forall\] Es una generalización de la \emph{conjunción}. Debido a esto es verdadero cuando todos los valores de \(x\) que pertenecen al Dominio de \(A\) son Verdaderos. Se denota: \(\forall x ; p(x)\) Se lee: ``Para Todo \(x\)'', ``Para cada \(x\)'', ``Todos (as) las \(x\)'', ``Todo \(x\)''.

Sea \(A=\left\{1,2,3,4,5\right\}\) y la función proposicional \(3x-2<12\) entonces las proposiciones

\begin{enumerate}
\def\labelenumi{\arabic{enumi}.}
\tightlist
\item
  \(\forall x\in A:3x-1<14\)
\item
  \(\exists\; x\in A:3x-2<12\)
\end{enumerate}

son falsa y verdadera respectivamente

\begin{definition}[Proposición universal]
\protect\hypertarget{def:universal2}{}{\label{def:universal2} \iffalse (Proposición universal) \fi{} }Una \emph{proposición universal} es aquella que está provista de un \emph{cuantificador universal}, y tiene la forma: \[\forall x\in A:p(x)\]
\end{definition}

Una \emph{proposición universal} es aquella que está provista de un \emph{cuantificador universal}, y tiene la forma: \[\forall x\in A:p(x)
\]

\begin{definition}[Proposición existencial]
\protect\hypertarget{def:existencial2}{}{\label{def:existencial2} \iffalse (Proposición existencial) \fi{} }Una \emph{proposición existencial} es aquella que está provista de un \emph{cuantificador existencial}, y tiene la forma: \[\exists x\in A:p(x)\]
\end{definition}

Una \emph{proposición existencial} es aquella que está provista de un \emph{cuantificador existencial}, y tiene la forma: \[\exists x\in A:p(x)\]
\#\#\# Negación de las proposiciones universal y existencial

Cambiando el cuantificador universal por el cuantificador existencial, o viceversa, es decir \[\sim[\exists x\in A; P(x)]\equiv\forall x\in A;\sim P(x)\]

\[
\sim\left[\forall  x\in A; P(x)\right]\equiv\exists\; x\in A;\sim P(x)
\]

La negación del \textbf{\emph{cuantificador universal}} es equivalente a la \emph{afirmación de un cuantificador existencial} respecto de la \textbf{\emph{función proposicional negada}.}

La negación de un \textbf{\emph{cuantificador existencial}} es equivalente a la \emph{afirmación de un cuantificador universal} respecto de la \textbf{\emph{función proposicional negada}}.

\begin{example}
\protect\hypertarget{exm:wwwwwwww}{}{\label{exm:wwwwwwww} }Dada la proposición: ``Si todos los números primos son impa­res, los números positivos son mayores que -1''

\begin{itemize}
\tightlist
\item
  Expresarla simbólicamente
\item
  Negar oracionalmente la proposición
\end{itemize}
\end{example}

\begin{solution}
\iffalse{} {Solución. } \fi{}
Sea \(p(x):\) números primos son impares y \(q(x):\) números positivos mayores que -1

\begin{itemize}
\item
  \(\forall x:[p(x)\rightarrow q(x)]\)
\item
  Negando el item anterior

  \[
  \begin{aligned}
  \sim\left\{\forall x:[p(x)\rightarrow q(x)]\right\}
  &\equiv \sim\left\{\forall x:p(x)\rightarrow \forall x:q(x)\right\}\\
  &=\sim\left\{\sim[\forall x:p(x)]\vee \forall x:q(x)\right\}\\
  &\equiv\forall x:p(x)\wedge \exists\; x:\sim q(x)
  \end{aligned}
  \]

  que se lee: ``Todos los números primos son impares y algunos números no son mayores que -1''
\end{itemize}
\end{solution}

Sea \(p(x):\) números primos son impares y \(q(x):\) números positivos mayores que -1

\begin{itemize}
\item
  \(\forall x:[p(x)\rightarrow q(x)]\)
\item
  Negando el item anterior

  \[
  \begin{aligned}
  \sim\left\{\forall x:[p(x)\rightarrow q(x)]\right\}
  &\equiv \sim\left\{\forall x:p(x)\rightarrow \forall x:q(x)\right\}\\
  &=\sim\left\{\sim[\forall x:p(x)]\vee \forall x:q(x)\right\}\\
  &\equiv\forall x:p(x)\wedge \exists\; x:\sim q(x)
  \end{aligned}
  \]

  que se lee: ``Todos los números primos son impares y algunos números no son mayores que -1''
\end{itemize}

\begin{example}
\protect\hypertarget{exm:wwwwwwwu}{}{\label{exm:wwwwwwwu} }Dado el conjunto \(A=\left\{x\in\mathbb{N}:-14<x<27\right\}\). Hallar el valor de verdad de \[
s=[(\sim p\wedge \sim q)\rightarrow(\sim q\wedge \sim r)]\leftrightarrow(\sim p\vee r) 
\] si \(p=(\forall x\in A, \exists y\in A, \forall z\in A)[x^2-z^2>y^2]\), \(q=(\exists y\in A, \forall z\in A, \exists x \in A)[2x-4y<-z]\) y \(r=(\forall z\in A, \exists x\in A, \forall y \in A)[3x^2-z^2>y]\)
\end{example}

\begin{solution}
\iffalse{} {Solución. } \fi{}\(A=\left\{1,2,3,\ldots,26\right\}\) luego el valor de \(\text{V}(p)=F\), \(\text{V}(q)=V\) y \(\text{V}(r)=V\) pues

\begin{itemize}
\tightlist
\item
  Si \(y=1\) entonces \(x^2-z^2>y^2\equiv x^2>1+z^2\) lo cual no es valido \(\forall x,z\in A\) entonces \(\text{V}(p)=F\)
\item
  Si \(y=25\in A\) y \(x=1\in A\) entonces \(2x-4y<-z\equiv 2+z<100\) lo cual es valido \(\forall z\in A\) entonces \(\text{V}(q)=V\)
\item
  Si \(x=26\in A\) entonces \(3x^2-z^2>y\equiv3(26)^2>z^2+y\) lo cual es valido \(\forall z,y\in A\) entonces \(\text{V}(r)=V\)
\end{itemize}

por lo tanto

\[
\begin{aligned}
\text{V}(s)&=\text{V}[(\sim p\wedge \sim q)\Longrightarrow(\sim q\wedge \sim r)]\Longleftrightarrow(\sim p\vee r)\\
&=[(V\wedge F)\Longrightarrow(F\wedge F)]\Longleftrightarrow(V\vee V)\\
&=[F\Longrightarrow F]\Longleftrightarrow V\\
&=V
\end{aligned}
\]
\end{solution}

\begin{exercise}
\protect\hypertarget{exr:unnamed-chunk-3}{}{\label{exr:unnamed-chunk-3} }Dada la proposición: ``Obtendré un puntaje aprobatorio si y solo si estudio concienzudamente el curso''

\begin{itemize}
\tightlist
\item
  Expresarla simbólicamente
\item
  Negar oracionalmente la proposición
\end{itemize}
\end{exercise}

\begin{exercise}
\protect\hypertarget{exr:unnamed-chunk-4}{}{\label{exr:unnamed-chunk-4} }Dado el conjunto \(G=\left\{x\in\mathbb{Z}^+:-14<2x<20\right\}\). Hallar el valor de verdad de \[
s=(p\wedge \sim q)\rightarrow[(\sim q\wedge \sim r)\leftrightarrow(\sim p\vee r)] 
\] si \(p=(\forall x\in A, z\in \mathbb{N_0})[xz\in \mathbb{Z}]\), \(q=(\forall z\in A, \exists x \in A)[x\neq y]\) y \(r=(\forall z\in A, \forall y \in A)[yx^2>500]\)
\end{exercise}

\hypertarget{conjuntos-iguales}{%
\section{Conjuntos Iguales}\label{conjuntos-iguales}}

\[\begin{aligned}A=B&\Longleftrightarrow \left\{(x\in A\rightarrow x\in B)\wedge(x\in B\rightarrow x\in A)\right\}\\
&\Longleftrightarrow x\in A \leftrightarrow x\in B 
\end{aligned}\]

\[\begin{aligned}A\neq B&\Longleftrightarrow \left\{(\exists x\in A; x\notin B)\vee(\exists x\in B; x\notin A)\right\}\\
&\Longleftrightarrow x\in A \leftrightarrow x\in B 
\end{aligned}\]

\hypertarget{propiedades}{%
\subsection{Propiedades}\label{propiedades}}

\begin{itemize}
\tightlist
\item
  \(A=A\)
\item
  \(A=B\rightarrow B=A\)
\item
  \(A=B\) y \(B=C\) entonces \(A=C\)
\end{itemize}

\hypertarget{inclusiuxf3n-y-subconjuntos}{%
\section{Inclusión y subconjuntos}\label{inclusiuxf3n-y-subconjuntos}}

\[
\begin{aligned}
A\subset B&\leftrightarrow\left\{x\in A\rightarrow x\in B\right\}\\
&\leftrightarrow\left\{\forall x\in A, x\in B\right\}
\end{aligned}
\]

\[
A\not\subset B\leftrightarrow\exists x\in A\;|\; x\notin B
\]

\hypertarget{propiedades-1}{%
\subsection{Propiedades}\label{propiedades-1}}

\begin{itemize}
\tightlist
\item
  \(A\subset A\)
\item
  \(A\subset B\wedge B\subset A\rightarrow A\subset B\)
\item
  \(A\subset B\wedge B\subset C\rightarrow A\subset C\)
\item
  \(\forall A\) \(\emptyset\subset A\)
\end{itemize}

\hypertarget{conjuntos-disjuntos}{%
\section{Conjuntos disjuntos}\label{conjuntos-disjuntos}}

\[
A\text{ disjunto de } B\leftrightarrow\nexists x\;|\; x\in A\wedge x\in B  
\]

\hypertarget{conjunto-potencia}{%
\section{Conjunto potencia}\label{conjunto-potencia}}

\[
P(A)=\left\{X\;|\;X\subset A\right\}
\]

\begin{remark}
\iffalse{} {Observación. } \fi{}
* \(P(A)\) tiene \(2^n\) elementos
* \(\emptyset\in P(A)\)
* \(A\in P(A)\)
\end{remark}

Propiedades

\begin{itemize}
\tightlist
\item
  \(P(\emptyset)=\left\{\emptyset\right\}\)
\item
  \(A\subset B\leftrightarrow P(A)\subset P(B)\)
\item
  \(A= B\leftrightarrow P(A)= P(B)\)
\end{itemize}

\hypertarget{representaciuxf3n-gruxe1fica-de-los-conjuntos}{%
\section{Representación Gráfica de los Conjuntos}\label{representaciuxf3n-gruxe1fica-de-los-conjuntos}}

Diagrama de euler

\hypertarget{operaciones-entre-conjuntos}{%
\section{Operaciones entre conjuntos}\label{operaciones-entre-conjuntos}}

\hypertarget{uniuxf3n}{%
\subsection{Unión}\label{uniuxf3n}}

\[
A\cup B
=\left\{x/x\in A\vee x\in B\right\}
\]

Propiedades

\begin{itemize}
\tightlist
\item
  \(A\cup A=A\)
\item
  \(A\cup \emptyset=A\)
\item
  \(A\cup U=U\)
\item
  \(A\cup B=B\cup A\)
\item
  \((A\cup B)\cup C=A\cup(B\cup C)\)
\end{itemize}

\hypertarget{intersecciuxf3n}{%
\subsection{Intersección}\label{intersecciuxf3n}}

\[
A\cap B=\left\{x/x\in A\wedge x\in B\right\}
\]

\[
x\in A\cap B\leftrightarrow x\in A\wedge x\in B
\]

Propiedades

\begin{itemize}
\tightlist
\item
  \(A\cap A=A\)
\item
  \(A\cap \emptyset=\emptyset\)
\item
  \(A\cap U=A\)
\item
  \(A\cap B=B\cap A\)
\item
  \((A\cap B)\cap C=A\cap(B\cap C)\)
\end{itemize}

\hypertarget{diferencia}{%
\subsection{Diferencia}\label{diferencia}}

\[
A- B=\left\{x/x\in A\wedge x\notin B\right\}
\]

\[
x\in A- B\leftrightarrow x\in A\wedge x\notin B
\]

Propiedades

\begin{itemize}
\tightlist
\item
  \(A- A=\emptyset\)
\item
  \(A- \emptyset=A\)
\item
  \(\emptyset-A=\emptyset\)
\item
  \(A- B\subset A\)
\item
  \((A-B)=(A\cup B)-B)=A-(A\cap B)\)
\end{itemize}

\hypertarget{complemento}{%
\subsection{Complemento}\label{complemento}}

\[
\mathcal{C}_BA=B-A=\left\{x/x\in B\wedge x\notin A\right\}
\]

\[
x\in \mathcal{C}_BA\leftrightarrow x\in B\vee x\notin W
\]

Si \(B=U\) entonces \(\mathcal{C}_BA=A'=A^C=\overline{A}\)

Propiedades

\begin{itemize}
\tightlist
\item
  \(\mathcal{C}_BA\subset B\) y \(\mathcal{C}_AB\subset A\)
\item
  \(A'\cup A=U\) o \(A\cup \mathcal{C}_AB=A\)
\item
  \(A\cap A'=\emptyset\) o \(A\cap \mathcal{C}_AB=\emptyset\)
\item
  \(U'=\emptyset\) o \(\mathcal{C}_AA=\emptyset\)
\item
  \(\emptyset'=U\) o \(\mathcal{C}_A\emptyset=A\)
\item
  \((A')'=A\) o \(\mathcal{C}_B(\mathcal{C}_BA)=A\)
\item
  \(A-B=A\cap B'\) o \(A-B=A\cap \mathcal{C}_AB\)
\end{itemize}

\hypertarget{diferencia-simuxe9trica}{%
\subsection{Diferencia simétrica}\label{diferencia-simuxe9trica}}

\[
A\Delta B=\left\{x/(x\in A\wedge x\in B)\vee (x\in A\wedge x\in B)\right\}
\]

\[
x\in A\Delta B\leftrightarrow (x\in A\wedge x\in B)\vee (x\in A\wedge x\in B)
\]

Propiedades

\begin{itemize}
\tightlist
\item
  \(A\Delta B=\emptyset\)
\item
  \(A\Delta \emptyset=A\)
\item
  \(A\Delta B=B\Delta A\)
\item
  \((A\Delta B)\Delta C=A\Delta(B\Delta C)\)
\item
  \((A\Delta B)\cap C=(A\Delta C)\Delta(B\Delta C)\)
\item
  \((A\Delta B)\cup(B\Delta C)=(A\cup B\cup C)-(A\cap B\cap C)\)
\end{itemize}

\hypertarget{ejercicios}{%
\subsection{Ejercicios}\label{ejercicios}}

\begin{enumerate}
\def\labelenumi{\arabic{enumi}.}
\item
  Sea \(U=\left\{x\in\mathbb{N}|0<x\leq 10\right\}\) y los subconjuntos: \(A=\left\{x\in\mathbb{N}|x \text{ es primo}\right\}\), \(B=\left\{x\in\mathbb{N}| x\text{ es es un cuadrado perfecto}\right\}\) y \(C=\left\{x\in\mathbb{N}|x\text{ es impar}\right\}\). Hallar

  \begin{itemize}
  \tightlist
  \item
    \((A\cup B)'-C\)
  \item
    \((A-C)'\cap B\)
  \item
    \((A\Delta B)-(A\Delta C)\)
  \item
    \((A\cap C)'-(B\cup C)'\)
  \end{itemize}
\item
  Dados los conjuntos \(A=\left\{x\in\mathbb{Z}|\sim[x\leq -2\vee x>3]\right\}\), \(B=\left\{x\in\mathbb{N}|\sim[-1<x\leq 3 \rightarrow x=5]\right\}\) y \(C=\left\{x\in\mathbb{Z}|(x< -2\vee x\geq 2)\rightarrow x>1\right\}\) Hallar el resultado de \((B\cap C)\Delta(A\cap B)\)
\end{enumerate}

\hypertarget{ejercicios-1}{%
\subsection{Ejercicios}\label{ejercicios-1}}

\begin{enumerate}
\def\labelenumi{\arabic{enumi}.}
\tightlist
\item
  Sombree las regiones correspondientes a los conjuntos
\end{enumerate}

\begin{itemize}
\item
  \(\left\{ \left[ \left( A\cup B \right)'\cap \left( C \Delta D \right) \right] \cap B \right\} \Delta C\)
\item
  \(\left[ \left( A\cup B \right)'\cap \left( C \Delta D \right) \right]-\left( B\cap C \right)\)
\item
  \(\left\{ \left[ \left( A\cup B \right)'\cap C \right] \Delta D \right\} -\left( A\cup B \right)\)
\end{itemize}

\hypertarget{ejemplo}{%
\paragraph{Ejemplo}\label{ejemplo}}

\[ \left\{  \left[ \left( A\cup B \right)'\cap  C \right] \cap B \right\} \Delta C \]

\hypertarget{nuxfamero-de-elementos-de-un-conjunto.-propiedades}{%
\section{Número de elementos de un Conjunto. Propiedades}\label{nuxfamero-de-elementos-de-un-conjunto.-propiedades}}

\[
n(A)
\]

\hypertarget{funciones-y-relaciones}{%
\chapter{Funciones y relaciones}\label{funciones-y-relaciones}}

\hypertarget{numeros-reales}{%
\chapter{Numeros reales}\label{numeros-reales}}

\hypertarget{funciones-exponenciales-logaruxedtmicas}{%
\chapter{Funciones exponenciales logarítmicas}\label{funciones-exponenciales-logaruxedtmicas}}

\hypertarget{inducciuxf3n-matemuxe1tica}{%
\chapter{Inducción matemática}\label{inducciuxf3n-matemuxe1tica}}

\hypertarget{suceciones}{%
\chapter{Suceciones}\label{suceciones}}

\hypertarget{nuxfameros-complejos}{%
\chapter{Números complejos}\label{nuxfameros-complejos}}

\hypertarget{polinomios}{%
\chapter{Polinomios}\label{polinomios}}

\hypertarget{appendix-apendice}{%
\appendix \addcontentsline{toc}{chapter}{\appendixname}}


Temas de reforzamiento o conocimientos preliminares que son necesarias para entender el contenido.

\hypertarget{trasformaciones}{%
\chapter{Trasformaciones}\label{trasformaciones}}

\bibliography{book.bib,packages.bib}

\printindex

\end{document}
