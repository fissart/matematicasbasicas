% Options for packages loaded elsewhere
\PassOptionsToPackage{unicode}{hyperref}
\PassOptionsToPackage{hyphens}{url}
\PassOptionsToPackage{dvipsnames,svgnames*,x11names*}{xcolor}
%
\documentclass[
  16pt,
]{krantz}
\usepackage{amsmath,amssymb}
\usepackage{lmodern}
\usepackage{ifxetex,ifluatex}
\ifnum 0\ifxetex 1\fi\ifluatex 1\fi=0 % if pdftex
  \usepackage[T1]{fontenc}
  \usepackage[utf8]{inputenc}
  \usepackage{textcomp} % provide euro and other symbols
\else % if luatex or xetex
  \usepackage{unicode-math}
  \defaultfontfeatures{Scale=MatchLowercase}
  \defaultfontfeatures[\rmfamily]{Ligatures=TeX,Scale=1}
  \setmonofont[Scale=0.7]{Source Code Pro}
\fi
% Use upquote if available, for straight quotes in verbatim environments
\IfFileExists{upquote.sty}{\usepackage{upquote}}{}
\IfFileExists{microtype.sty}{% use microtype if available
  \usepackage[]{microtype}
  \UseMicrotypeSet[protrusion]{basicmath} % disable protrusion for tt fonts
}{}
\makeatletter
\@ifundefined{KOMAClassName}{% if non-KOMA class
  \IfFileExists{parskip.sty}{%
    \usepackage{parskip}
  }{% else
    \setlength{\parindent}{0pt}
    \setlength{\parskip}{6pt plus 2pt minus 1pt}}
}{% if KOMA class
  \KOMAoptions{parskip=half}}
\makeatother
\usepackage{xcolor}
\IfFileExists{xurl.sty}{\usepackage{xurl}}{} % add URL line breaks if available
\IfFileExists{bookmark.sty}{\usepackage{bookmark}}{\usepackage{hyperref}}
\hypersetup{
  pdftitle={Matemáticas básicas},
  pdfauthor={Ricardo Michel MALLQUI BAÑOS},
  colorlinks=true,
  linkcolor=Maroon,
  filecolor=Maroon,
  citecolor=Blue,
  urlcolor=Blue,
  pdfcreator={LaTeX via pandoc}}
\urlstyle{same} % disable monospaced font for URLs
\usepackage{longtable,booktabs,array}
\usepackage{calc} % for calculating minipage widths
% Correct order of tables after \paragraph or \subparagraph
\usepackage{etoolbox}
\makeatletter
\patchcmd\longtable{\par}{\if@noskipsec\mbox{}\fi\par}{}{}
\makeatother
% Allow footnotes in longtable head/foot
\IfFileExists{footnotehyper.sty}{\usepackage{footnotehyper}}{\usepackage{footnote}}
\makesavenoteenv{longtable}
\setlength{\emergencystretch}{3em} % prevent overfull lines
\providecommand{\tightlist}{%
  \setlength{\itemsep}{0pt}\setlength{\parskip}{0pt}}
\setcounter{secnumdepth}{5}
\usepackage[spanish,es-lcroman]{babel}
\usepackage{booktabs}
\usepackage{mathtools}
\usepackage{graphicx}
\usepackage{amsmath}
\usepackage{makeidx}
\makeindex
%\usepackage{showframe}
%\usepackage[a4paper]{geometry}
%\geometry{verbose,tmargin=3cm,bmargin=3cm,lmargin=3.5cm,rmargin=3cm}
\setlength\parindent{23pt}
\usepackage[singlelinecheck=off]{caption}

\usepackage{times}
\renewcommand{\rmdefault}{ptm}
%\usepackage[lite,subscriptcorrection,nofontinfo,zswash]{mtpro2}

\usepackage{graphicx}

% Determine if the image is too wide for the page.
\makeatletter
\def\ScaleIfNeeded{%
  \ifdim\Gin@nat@width>\linewidth
    \linewidth
  \else
    \Gin@nat@width
  \fi
}
\makeatother

% Resize figures that are too wide for the page.
%\let\oldincludegraphics\includegraphics
%\renewcommand\includegraphics[2][]{%
%  \oldincludegraphics[scale=0.85]{#2}
%}

\usepackage{amsthm}
\makeatletter
\def\thm@space@setup{%
  \thm@preskip=8pt plus 2pt minus 4pt
  \thm@postskip=\thm@preskip
}
\makeatother

\flushbottom 

\frontmatter

\ifluatex
  \usepackage{selnolig}  % disable illegal ligatures
\fi
\usepackage[]{natbib}
\bibliographystyle{apalike}

\title{Matemáticas básicas}
\author{Ricardo Michel MALLQUI BAÑOS}
\date{2021-06-16}

\usepackage{amsthm}
\newtheorem{theorem}{Teorema}[chapter]
\newtheorem{lemma}{Lema}[chapter]
\newtheorem{corollary}{Corolario}[chapter]
\newtheorem{proposition}{Proposición}[chapter]
\newtheorem{conjecture}{Conjectura}[chapter]
\theoremstyle{definition}
\newtheorem{definition}{Definición}[chapter]
\theoremstyle{definition}
\newtheorem{example}{Ejemplo}[chapter]
\theoremstyle{definition}
\newtheorem{exercise}{Ejercicio}[chapter]
\theoremstyle{definition}
\newtheorem{hypothesis}{Hypothesis}[chapter]
\theoremstyle{remark}
\newtheorem*{remark}{Observación}
\newtheorem*{solution}{Solución}
\begin{document}
\maketitle

%\cleardoublepage\newpage\thispagestyle{empty}\null
%\cleardoublepage\newpage\thispagestyle{empty}\null
%\cleardoublepage\newpage
\thispagestyle{empty}
\begin{flushright}
Universidad Nacional San Cristobal de Huamanga

Fisart.cf

Agradecimento a los estudiantes de la ESFAPA FGPA

A la UNSCH

\includegraphics[height=3cm]{U.pdf}
\end{flushright}
%\setlength{\abovedisplayskip}{-5pt}
%\setlength{\abovedisplayshortskip}{-5pt}

{
\hypersetup{linkcolor=}
\setcounter{tocdepth}{2}
\tableofcontents
}
\listoftables
\listoffigures
\newcommand{\N}{\mathbb{N}}
\newcommand{\R}{\mathbb{R}}
\newcommand{\CC}{\mathbb{C}}
\newcommand{\I}{\mathbb{I}}
\newcommand{\f}{\mathbb{f}}
\newcommand{\X}{\mathbb{X}}
\newcommand{\D}{\mathbb{D}}
\newcommand{\Z}{\mathbb{Z}}
\newcommand{\Q}{\mathbb{Q}}
\newcommand{\norm}[1]{\left\Vert#1\right\Vert}
\newcommand{\abs}[1]{\left\vert#1\right\vert}
\newcommand{\set}[1]{\left\{#1\right\}}
\newcommand{\seq}[1]{\left<#1\right>}
\newcommand{\co}[1]{\left[#1\right]}
\newcommand{\cc}[1]{\left(#1\right)}
\newcommand{\J}{\mathcal{J}}
\newcommand{\K}{\mathcal{K}}
\newcommand{\M}{\mathcal{M}}
\newcommand{\F}{\mathcal{F}}

\hypertarget{resumen}{%
\chapter*{Resumen}\label{resumen}}


www.

\hypertarget{introducciuxf3n}{%
\chapter*{Introducción}\label{introducciuxf3n}}


www.

\mainmatter

\hypertarget{logica}{%
\chapter{Logica}\label{logica}}

www.

\hypertarget{conjuntos}{%
\chapter{Conjuntos}\label{conjuntos}}

\begin{definition}[Conjunto]
\protect\hypertarget{def:conjunto}{}{\label{def:conjunto} \iffalse (Conjunto) \fi{} }ES una coleccion de elementos con caractersiticas similares
\end{definition}

\begin{definition}[Determinacion de conjuntos]
\protect\hypertarget{def:conjunto2}{}{\label{def:conjunto2} \iffalse (Determinacion de conjuntos) \fi{} }Por extencion y comprension
\end{definition}

\hypertarget{funciuxf3n-proposicional-y-cuantificadores}{%
\section{Función proposicional y cuantificadores}\label{funciuxf3n-proposicional-y-cuantificadores}}

\begin{definition}[existencial]
\protect\hypertarget{def:existencial}{}{\label{def:existencial} \iffalse (existencial) \fi{} }ES una coleccion de elementos con caractersiticas similares
\end{definition}

\[
\exists
\]

\begin{definition}[Universal]
\protect\hypertarget{def:universal}{}{\label{def:universal} \iffalse (Universal) \fi{} }ES una coleccion de elementos con caractersiticas similares
\end{definition}

\[
\forall
\]

\hypertarget{negaciuxf3n-de-los-cuantificadores}{%
\subsection{Negación de los cuantificadores}\label{negaciuxf3n-de-los-cuantificadores}}

\hypertarget{operaciones-entre-conjuntos}{%
\section{Operaciones entre conjuntos}\label{operaciones-entre-conjuntos}}

\hypertarget{relaciones-entre-conjuntos-conjuntos-iguales.-conjuntos-equivalentes}{%
\section{Relaciones entre Conjuntos: Conjuntos Iguales. Conjuntos equivalentes}\label{relaciones-entre-conjuntos-conjuntos-iguales.-conjuntos-equivalentes}}

\hypertarget{representaciuxf3n-gruxe1fica-de-los-conjuntos}{%
\section{Representación Gráfica de los Conjuntos}\label{representaciuxf3n-gruxe1fica-de-los-conjuntos}}

\hypertarget{uniuxf3n-de-conjuntos.}{%
\section{Unión de Conjuntos.}\label{uniuxf3n-de-conjuntos.}}

\hypertarget{intersecciuxf3n-de-conjuntos.-propiedades}{%
\section{Intersección de Conjuntos. Propiedades}\label{intersecciuxf3n-de-conjuntos.-propiedades}}

\hypertarget{distributivas-de-la-uniuxf3n-e-intersecciuxf3n}{%
\section{Distributivas de la Unión e Intersección}\label{distributivas-de-la-uniuxf3n-e-intersecciuxf3n}}

\hypertarget{leyes-de-absorciuxf3n}{%
\section{Leyes de Absorción}\label{leyes-de-absorciuxf3n}}

\hypertarget{diferencia-de-conjuntos.}{%
\section{Diferencia de Conjuntos.}\label{diferencia-de-conjuntos.}}

\hypertarget{complemento-de-un-conjunto.-propiedades}{%
\section{Complemento de un Conjunto. Propiedades}\label{complemento-de-un-conjunto.-propiedades}}

\hypertarget{diferencia-simuxe9trica.}{%
\section{Diferencia Simétrica.}\label{diferencia-simuxe9trica.}}

\hypertarget{nuxfamero-de-elementos-de-un-conjunto.-propiedades}{%
\section{Número de elementos de un Conjunto. Propiedades}\label{nuxfamero-de-elementos-de-un-conjunto.-propiedades}}

\hypertarget{funciones-y-relaciones}{%
\chapter{Funciones y relaciones}\label{funciones-y-relaciones}}

\hypertarget{numeros-reales}{%
\chapter{Numeros reales}\label{numeros-reales}}

\hypertarget{funciones-exponenciales-logaruxedtmicas}{%
\chapter{Funciones exponenciales logarítmicas}\label{funciones-exponenciales-logaruxedtmicas}}

\hypertarget{inducciuxf3n-matemuxe1tica}{%
\chapter{Inducción matemática}\label{inducciuxf3n-matemuxe1tica}}

\hypertarget{suceciones}{%
\chapter{Suceciones}\label{suceciones}}

\hypertarget{nuxfameros-complejos}{%
\chapter{Números complejos}\label{nuxfameros-complejos}}

\hypertarget{polinomios}{%
\chapter{Polinomios}\label{polinomios}}

\hypertarget{appendix-apendice}{%
\appendix \addcontentsline{toc}{chapter}{\appendixname}}


Temas de reforzamiento o conocimientos preliminares que son necesarias para entender el contenido.

\hypertarget{trasformaciones}{%
\chapter{Trasformaciones}\label{trasformaciones}}

  \bibliography{book.bib,packages.bib}

\printindex

\end{document}
